\documentclass[12pt]{article}
\usepackage[a4paper]{geometry}
\usepackage{natbib}
\usepackage{graphicx}
\usepackage[parfill]{parskip}
\usepackage{float}

\usepackage{ifxetex,ifluatex}

\ifnum 0\ifxetex 1\fi\ifluatex 1\fi=0 % if pdftex
  \usepackage[T1]{fontenc}
  \usepackage[utf8]{inputenc}
  \usepackage{textcomp} % provides euro and other symbols
\else % if luatex or xelatex
  \usepackage{unicode-math}
  \defaultfontfeatures{Ligatures=TeX,Scale=MatchLowercase}
\fi


\title{GET - Rapport- CandyFeine}
\author{Joel Schar, Yann Lederrey, Steve Henriquet, Anthoine Nourazar }
\date{December 2018}



\begin{document}

\maketitle

\section{Introduction}
Le mode de vie de notre société actuelle est devenu plus fatiguant, plus stressant, et plus contraignant. C'est la raison pour laquelle, de plus en plus de produits arrivent sur le marché afin d'aider les personnes à maintenir ce rythme de vie. La solution que nous offrons est bien différente de celles proposées. En effet, il existe bon nombre de boissons énergisante ou de barres énergiques pour offrir un boost au cours de la journée. Mais ce que nous proposons est à part. Avec les bonbons CandyFein, nous vous permettons de déguster de savoureux bonbons qui vous procurerons de l'énergie tout au long de la journée.

\section{Business Modèle}
\includegraphic{../img/BusinessModel.jpg}

\section{Structure juridique}
Nous choisissons de faire une Société à responsabilité limitée (Sarl)
Cela car nous serions un petite entreprise et que le capital initial est raisonnable : 20 milles CHF.

\section{Diagnostic stratégique de votre activité}
\subsection(Analyse externe) 
\textbf PESTEL


\subsection Politique
Activités de lobbying
Délocalisation
Politique Europeenne

\subsection Economique
Evolution du pouvoir d'achat
Confiance des consommateurs
Taux d'inflation

\subsection Social
Système de la santé
Sécurité des personnes et des biens
Conditions de vie
Style de vie, mode et tendances
Accès à l'information, internet et réseaux sociaux

\subsection Technologique
Recherche & développement
Aides au financement de la recherche et de l’innovation
Découvertes R&D
Nouveaux brevets

\subsection Ecologique
Météo et climat
Mobilité et transports

\subsection Legal
Réglementation des marchés
Protection des consommateurs

voir annexe


\subsection Analyse interne

\section{Publique Cible}
Le bonbon "CandyFein" est un bonbon qui se veut énergisant. Il sera consommé au ressenti d'une baisse d'energie ou d'un besoin d'attention accrue. En début de jounrée, après un bon repas, en fin de journée ou début de soirée ce tout des petit moment ou un petit coup de fouet avant de repartir peut s'avérer nécessaire.\\
Ce type de besoin est plutôt typique d'une population active qui à une journée bien remplie. Un publique jeune qui ne veut pas forcément prendre le temps de s'arrêter pour boire un café et qui préfère gagner du temps en mangeant une petite dose d'énergie, délicieusement sucrée, en même temps qu'il fait autre chose.\\
Ce produit parfaitement dosé saura trouver sa place dans la poche d'un blouson et sera facilement accessible en tout temps.\\

\section{Le Produit}
CandyFeine
\subsection{Caractéristiques du produit}
CandyFeine sont des bonbons pétillants gelifiés en forme de grains de café comprenant de la cafeine et à de différents gouts.
Vendu sous forme de paquets de 10 ou 20 bonbons.
\subsection{Assortiment, gamme}
\textbf{Différents gout de bonbons :}

\textit{* : signifie que c'est encore en prototype}
\begin{itemize}
 \item Pomme-citron avec sucre pétillant
 \item Citron avec sucre pétillant
 \item framboise avec sucre pétillant
 \item café *
\end{itemize}

\textbf{Différents produits énergisant :}
\begin{itemize}
 \item Caféine
 \item Guarana *
\end{itemize}

\subsection{Ingrédients}
\textbf{Correspond à la préparation de 180 grammes de bonbons pétillant pomme-citron}
\textbf{ingérdients des bonbons}

\textit{Certaines recettes peuvent contenir du colorant alimentaire naturels}
\begin{itemize}
 \item 100ml de jus de pomme
 \item 100g de sirop de glucide
 \item 8g d'amidion
 \item 8g d'agar agar
 \item 4g d'extrait de citron
 \item 135mg de caféine pure
\end{itemize}

\textbf{ingrédients du sucre acidulé}
\begin{itemize}
 \item 6g d'acide citrique
 \item 18g de sucre fin
\end{itemize}

\subsection{Qualité}
Produit de manière artisanale dans le respect des fournisseurs et des consommateur.
CandyFeine joue sur la transparence de ses produits et leur provenance.
Nous assurons des produits de qualité, le plus local possible. Nos bonbons sont vegans.
\subsection{Marque}
Au sein de l'entreprise CandyFeine Sàrl notre marque CandyFeine décrit le produit en un mot, candy pour le bonbon et Feine en rapport à la cafeine contenue à l'intérieur.
Le nom de la marque permet donc de rapidement assimiler le nom à un type de produit et l'anglisisme permet plus facielement l'exportation des bonbons.

\subsection{Accessoires}
Dans l'état actuelle des choses nous n'avons pas jugé utiles la création d'accessoires.

\subsection{Packaging}
Actuellement nous proposons nos bonbons par paquet de 10 ou 20, Ceci correspond à la dose d'une ou deux boisson energisantes connue du marché actuel.
Le packaging est sous-forme de sachet avec une étiquette décrivant les ingrédients des bonbons.

\begin{figure}[H]
\centering
   \caption{\label{étiquette} étiquette des CandyFeine gout café}
   \includegraphics[scale=0.8]{../img/designEtiquettes_Cafe.png}
\end{figure}

\begin{figure}[H]
\centering
   \caption{\label{étiquette} étiquette des CandyFeine gout citron}
   \includegraphics[scale=0.8]{../img/designEtiquettes_Citron.png}
\end{figure}

\begin{figure}[H]
\centering
   \caption{\label{étiquette} étiquette des CandyFeine gout framboise}
   \includegraphics[scale=0.8]{../img/designEtiquettes_Framboise.png}
\end{figure}

\begin{figure}[H]
\centering
   \caption{\label{étiquette} étiquette des CandyFeine gout pomme-citron}
   \includegraphics[scale=0.8]{../img/designEtiquettes_pommeCitron.png}
\end{figure}


\subsection{Prestations annexes}
A l'heure actuelle nous proposants uniquement le conseil client comme préstation annexe.

\section{Modèle du Produit}

\section{Processus}
\includegraphics[scale=0.4]{../processus_de_commande.png} 

Le processus décrit ci dessus illustre le flux que nous envisageons pour la production de bonbons suite à une commande. Le délais pour ce processus dépend comme décrit du stocke de bonbons déjà préparés et de la quantité d'ingrédients qu'il faudra recommander.\\
On peut envisager les délais d'attentes suivant:\\
\begin{itemize}
\item délais de livraison des ingrédients : ~ 3 jours
\item temps de repos :  5 jours
\end{itemize}
On peut donc envisager qu'avec le temps de préparation des bons et l'envoie le délais maximum pour une commande sera de 2 semaines.

Ce processus est orienté pour la commande depuis notre site internet. En effet nous ne prenons ici pas en compte de processus qui pourrait arriver lors de vente ponctuelle sur un marché ou dans un foire. Ces événements étant exceptionnels et non prévisibles nous ne les avons pas inclus et sommes resté sur un processus de base simple.

\section{Prix}
Le prix de la matière première pour une quantité de 1600 portions de bonbons nous revient à 560.- . Sur cette base on peut conclure que le prix de la matière première pour une portion revient à 0.35 CHF.\\
A cela viennent s'ajouter le prix de l’emballage de 0.10 CHF.\\

En ajoutant à cela les frais fixe pour la production ainsi que les salaires des employés et en faisant une corrélation des ces coûts avec les capacité de production, on trouve que nous devons vendre notre produit à un prix raisonnable de 4.50 CHF.

Ingrédients ( matière première ) : 0.35 CHF \\
Embalage : 0.10 CHF \\
Prix final du produit :  4.50 CHF\\

Cette différence de prix nous permet de couvrir les frais de production, d'amortir l'investissement et de dégager un salaire correct pour les employés. Selon la courbes prévisionnelle sur 5 ans on voit que cette marges reste minime au début, mais permet ensuite de dégager un joli bénéfice qui devrait permettre de réinvestir et d'améliorer la structure pour faire tourner l'entreprise.

\section{Objectifs commerciaux}
\subsection{Stratégie d'attirance des clients}
Notre produits peut toucher un vaste pannel de personne nous pouvons tout de meme
nous centrer sur le monde étudiants qui sont de grand consommateurs de caféine, ainsi que les routiers ou encore les sportifs.\\
L'idée est donc de commencer la publicité auprès des université et des stations services. Par la suite, on peut imaginer aller à des
foire ou passer sur de la publicité de plus grande échelle tel que radio, affiches publicitaires. La  vente se déroulant aussi en ligne l'utilisation de GoogleAds est aussi intéressante à mettre en place.
\subsection{preuves de traction}
Actuellement seul le monde estudiantin a été interpellé par rapport à notre produit. Ces derniers se trouvent etre intéressé par l'apport en caféine du produit.
\subsection{cout d'acquisition des clients}
Dans le cas ou on commence la publicité par l'approche des étudiants et de stations services, il faut compter le cout en trajet, en affiche A4 publicitaire ainsi qu'en rabais sur les bonbons afin d'attirer les clients.

\subsection{expension du nombre de clients}
Par la suite, il faudrait réussir à faire notre place face aux produits energétiques existants. Pour cela le bouche à oreille et le buzz est notre plus grande chance.\\
En cas de buzz, on peut donc imaginer une augmantation rapide et exponentielle du nombre d'acheteurs.\\
Dans l'autre cas, il faudrait voir une augmantation plus stable et lent des consommateurs.

\section{Publicité}
La publicité va principalement s'appuyer sur le bouche à oreilles via les foires et présentoires que nous allons installer pendant de courtes périodes dans certaines villes Suisses pour promouvoir le produit et puisque nous faisons aussi de la vente en ligne, l'utilisation de Google Ads peut nous permettre une augmentations de visibilité online. Pour notre budget de 200CHF par mois, les Google Ads nous offrent une visibilité entre 3 745 et 6 246 de vue par mois et un nombre de clic prévisionelle entre 737 à 1 229 clics par mois sur la Suisse. 
\includegraphics[scale=0.4]{../img/googleAdsPrevision.jpg} 
Cela peut, même si tous les utilisateurs n'achètent pas, augmenter la visibilité des prochaines foires et stands où l'on sera présent. Pour les gens sceptiques par rapport au produit et/ou qui n'aime pas acheter en ligne, cela leur permettra de pouvoir goûter du produit via des sachet de 10g distribués et d'en acheter en direct.
L'utilisation d'une page Facebook et d'un compte Twitter est une option gratuite qui permet d'augmenter la visibilité et la présence du produits en ligne avec comme seul coût le temps qu'on lui accorde. 

\section{Budget prévisionnel sur 3-5 ans}
Le budget prévisionnel est basé sur une approche plutôt optimiste.
Et avec un apport initial de CHF 30 000.- répartis entre les quatres fondateurs. 

Il faudrait pour être en balance parfaite vendre plus de 1600 paquets de bonbons au prix fixé de 3,50 par paquets. Ce chiffre étant relativement élevé, les calculs ont été effectués dans un premier temps pour tabler sur une production et une vente de 800 paquets le premier mois. En plus de cela, il faut compter la production d'une quantité de 2,4kg de bonbons afin d'avoir 240 paquets de 10g d'échantillons au vue d'une distribution gratuite de produit. Pour ensuite dans les mois suivants atteindre une production de 1600 sachets et chaque demi-année, cette production et cette vente doublerait de la moitié de cette valeur de 1600 paquets ce qui aurait pour impact un doublement de la production de bonbon chaque année. Pour la part des échantillons, une valeur arbitraire équivalent à un trente-deuxième de la production sera ensuite produite car avec la croissance de l'entreprise, il faudra voir à s'exporter sur les pays voisins pour vendre des produits. Les échantillons seront les ambassadeurs toujours via des stands et des foires d'où l'ajout de frais de voyages ainsi que l'augmentation des fonds pour les foires.

Ensuite, il a été prévu qu'en cas de succès de l'entreprise, tous les employés augmentent leur temps de travail dans cette entreprise. Et que l'on engage même de nouveau employé. A partir d'un moment, nous avons jugés que la production serait tel que nous n'aurions plus d'autres choix que de louer un endroit pour pouvoir continuer la production donc mi-2021, il est prévu de cette acquisition. Cela a pour impact une augmentation dans les dépenses, d'abord le loyer, ensuite toutes les charges associés à ce nouvelle endroit et enfin le nouvelle investissement dans du matériel.

Pour ce qui est des salaires, nous avons prévu un salaire de CHF 70 000.- qui serait fixe pendant les deux premières années mais avec un temps de travail qui augmente au fur des années. Les salaires subissent après deux ans une augmentations de 10 000 pour ensuite gagner une augmentation de 15 000 par années. La première augmentation est plus faible pour compenser l'acquisition d'un nouveau local ainsi que l'arrivée d'un nouvelle employé.

L'augmentation général qui se produit se base sur le but de s'étendre avec des valeurs qui sembles cohérentes avec les possibilitées envisageables. Cependant, vu qu'aucune étude de marché n'a été réalisée certaines des valeurs sont purement arbitraires ainsi que les augmentations de prix liées à celle-ci.   

\section{Conclusion}
La mise sur le marché de notre produit nous semble prometteur. Nous avons des gens intéressés par le produit ce qui annonce des clients potentiels qui pourraient potentiellement acheter le produit voir devenir des clients régulier. Les bilans provisionnels que nous avons fait annoncent que nous devrions arriver à dégager un bénéfice suffisant dans les premiers mois pour faire démarrer l'entreprise et par la suite avoir des ressources suffisante pour réinvestir dans de nouvelle ressource de production. Il serait alors possible d'acheter des machines plus performantes et produire plus en moins de temps ou d'engager du personnels supplémentaire.

En ce qui concerne le cheminement du projet, nous avons remarqué que de prévoir et d'établir un cahier des charges réaliste est relativement difficile. Du fait que nous ne savons pas ce qui va marché et ce qui pourrait nous permettre de booster notre entreprise tout en ayant une gestion des coûts optimum et ainsi dégager le bénéfice maximum qui permettra de faire fructifier l'affaire.

Comme nous l'a dit M. Bruyndonckx, il n'existe pas de recette miracle et le seul moyen de réussir à lancer son business, c'est de faire au mieux et de se lancer. Si le début fonctionne suffisamment bien il sera ensuite possible de s'entourer des personnes qui permettrons de monter une structure solide et de pérenniser notre activité.

\bibliographystyle{plain}
\bibliography{références}
\end{document}
